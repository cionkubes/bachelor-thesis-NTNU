\chapter{Introduction}
\label{chap:introduction}
\section{Motivation}
The TIP-team at Trondheim municipality had a desire for a more agile development process and wanted a continuous deployment setup for their applications. Though because of security concerns they decided that none of the solutions that were offered for- and compatible with their current environment were sensible to implement. This is where the idea for a self-hosted continuous deployment solution came from.

The need for the cion product came from security concerns raised by the TIP-team at Trondheim municipality. They have been users of Continuous Integration, and to some degree \acrfull{CD} for a while, but as the scope of their work grew, so did the need for added security and functionality. After careful review they came to the conclusion that the current solution of using bitbucket pipelines could not be expanded to supply \acrshort{CD} without introducing security risks. Bitbucket pipelines requires the user to expose a management API out to the open internet, which raises several security concerns that will be explored in detail in this document.

\section{Research and Hypothesis}
\begin{tabularx}{\linewidth}{X X}
    \epigraph{Developers and businesses avoid \acrshort{CD} out of fear for applications hosted outside their control.}{\textit{Research problem}}
    & \epigraph{Which, if any, security benefits arise from self-hosted \acrshort{CD} solutions, as opposed to cloud-based solutions?}{\textit{Research question}}
\end{tabularx}
As a result of using a self-hosted \acrshort{CD} solution the authors expect to see a reduced risk associated with use of \acrshort{CD}. It is possible that services that would be facing the internet in a cloud-based solution could be internal in a self-hosted solution. If so, each service available to the internet in all likelihood introduces multiple attack vectors to either the host or the user.

It is also speculated that some of the services facing the internet would need control of high risk infrastructure. A cloud-based \acrshort{CD} solution needs the ability to launch and restart applications running in critical environments, in the case of \acrshort{CD} even the production environment. If this is the case, trusting the cloud-solution, here a foreign entity, would pose a risk that any security aware user or group should be wary of. Even if the foreign entity is trustworthy they might be a competitor or otherwise unfit to have control over important infrastructure.

\section{Report Structure}
The report is structured into six main chapters that cover larger subjects. The chapters are further divided into multiple subsections. The six main chapters are the introduction, theory, technology \& method, results, discussion, and the conclusion \& future work

\textbf{Introduction} introduces the research question and explains the rest of the report.

\textbf{Theory} presents the theoretical background that we base our decisions and work on. 

\textbf{Technology and Method} explains the technologies used in the software project. 

\textbf{Scientific method} explains the methods used in the scientific research conducted in this project. 

\textbf{Results} presents the scientific, engineering and administrative results.

\textbf{Discussion} reflects on the results presented in the Results chapter. 

\textbf{Conclusion and Future Work} reflects on how much of the  original vision was actually achieved as well as presents features and changes that could be implemented in the future.

\section{Acronyms and Glossary}
Glossary and acronyms are listed in appendix \ref{glos} and additional technical terms \& explanations are described in the Terms and Definitions document.
