\chapter{Introduction}
\label{chap:introduction}
The TIP-team at Trondheim municipality had a desire for a more agile development process and wanted a continuous deployment setup for their services. Though because of security concerns they decided that none of the solutions that were offered for- and compatible with their current environment were responsible to implement. This is why the idea for a self-hosted continuous deployment service was born.

The need for the cion product came from security concerns raised by the TIP-team at Trondheim municipality. They have been users of Continuous Integration, and to some degree Continuous Deployment and Delivery (CD) for a while, but as the scope of their work increased, so did the need for added security and functionality. After careful review they came to the conclusion that the current solution of using bitbucket pipelines could not be expanded to supply CD without introducing security risks. Bitbucket pipelines requires the user to expose a management API out to the open internet, and this raises several security concerns that (are described in part 3.5 of the vision document) (will be explored in this document?).

\section{Motivation}
The motivation for this project from the groups point of view came from the desire to work with containerisation and brand new technologies. Since containerisation is one of the big new buzzwords along with big data and machine learning this was a big opportunity to get acquainted with future technology. There is little focus on these new technologies in the current curriculum so the participants of this project felt this was a good way to familiarise themselves with technology they will probably end up using in the near future.

\section{Research and Hypothesis}
\begin{tabularx}{\linewidth}{X X}
    \epigraph{Developers and businesses avoid CD out of fear for applications hosted outside their control.}{\textit{Research problem}}
    & \epigraph{Which, if any, security benefits arise from self-hosted CD solutions, as opposed to cloud-based solutions?}{\textit{Research question}}
\end{tabularx}
As a result of using a self-hosted CD solution the authors expect to see a reduced risk associated with use of CD. It is possible that services that would be facing the internet in a cloud-based solution could be internal in a self-hosted solution. If so, each service available to the internet in all likelihood introduces multiple attack vectors to either the host or the user.

It is also speculated that some of the services facing the internet would necessarily need control of high risk infrastructure. A cloud-based CD solution needs the ability to launch and restart applications running in critical environments, in the case of CD even the production environment, from a foreign server. If this is the case trusting the foreign entity poses a risk any security aware user or group should be wary of. Even if the foreign entity is trustworthy they might be a competitor or otherwise unfit to assume control over important infrastructure.

\section{Report Structure}
The report is structured into six main chapters that cover larger subjects. The chapters are further divided into multiple subsections. The six main chapters are the introduction, theory, technology \& method, results, discussion, and the conclusion \& future work. The introduction serves as a way to introduce the research question and explain the rest of the report. In the theory part the aim is to present the theoretical background that we base our decisions and work on. In chapter three, Technology and Method, the technologies that are used in the software project are explained and the work process during development is described. In the the results chapter, the scientific, engineering and administrative results are presented. These results are then reflected upon in chapter 5 discussion. And finally for the Conclusion and Future Work chapter we will view how much of our vision we have actually achieved as well as presenting work that the project is currently missing and could be implemented in the future.

\section{Acronyms and Glossary}
Acronyms and Glossary is described in the Terms and Definitions document.
% referanse


  