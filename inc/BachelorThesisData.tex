\thesistitle{Self-hosted Continuous Delivery and Deployment to Docker Swarm and Kubernetes}
\thesisshorttitle{Self-hosted continuous delivery and deployment to Docker Swarm and Kubernetes} % use this if you have a very long title and want something shorter on the header pages
\thesisauthor{Harald Floor Wilhelmsen}
\thesisauthorA{Erlend Tobiassen}
\thesisauthorB{Kenan Mahic}
\thesissupervisor{Jan Harald Nilsen}

% There use to be a number associated with projects, this would help identify which project was selected.  If you are told to add a project number then this line adds the number.
\oppgaveNo{065}

\nmtkeywords{Thesis, Bachelor, Continuous Deployment, Continuous Delivery, Self-hosting, Docker, Docker Swarm, Kubernetes, Security, Software, CI, CD}
\nmtdesc{The purpose of this project is to develop a Continuous Deployment and Delivery tool that can be hosted in local service-environments. Additionally, the purpose is to examine the security concerns surrounding tools for Continuous Deployment and Delivery running in cloud-based solutions compared to self-hosted tools. Many of the current tools are hosted externally, so using them requires exposing the service-environment’s API to the externally hosted solution. This is a security-risk the contractor wishes to avoid. With this goal in mind a design that circumvents these security risks was developed. Based on this design it is concluded that a self-hosted solution would avoid most security-risks involved in opening a management API.}

\nmtoppdragsgiver{\NTNU}
\nmtcontact{Runar Andersstuen, 90635272, \newline runar.andersstuen@trondheim.kommune.no}

\thesisdate{\ntnubachelorthesisdate}
\useyear{05.06.2018}

\nmtappnumber{} %number of appendixes
\nmtpagecount{} %currently auto calculated but might be wrong


\thesistitleNOR{Egen-hosted Tjeneste for Continuous Delivery og Deployment mot Docker Swarm og Kubernetes}
\nmtkeywordsNOR{Avhandling, Bachelor, Continuous Deployment, Continuous Delivery, Lokal, Docker, Docker Swarm, Kubernetes, Sikkerhet, Software, CI, CD}
\nmtdescNOR{Formålet med denne oppgaven er å utvikle et hjelpeverktøy for Continuous Deployment og Continuous Delivery som er mulig å kjøre i eget tjenestemiljø. Mulige sikkerhetshull blir undersøkt både i eget-hostede løsninger og eksterne løsninger. Mange nåværende løsninger for Continuous Delivery og Deployment baserer seg på skytjenester. I slike løsninger må tjenestemiljøene åpnes til styring fra internett, dette er en sikkerhetsrisiko som oppgavestiller ikke ønsket å ta. For å nå formålet ble det designet en løsning som unngår sikkerhetsrisikoene assosiert med styring fra internett. Basert på dette designet konkluderers det med at en løsning som kan kjøres på lokale tjenestemiljø eliminerer mange av sikkerhetsrisikoene tilhørende åpning mot internett.}
