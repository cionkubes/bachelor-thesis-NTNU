\chapter{Technology and Method}
\label{chap:process}

\section{Technologies}
Most of the technologies were already selected and in use by the end of the fall project. Only the most prominent technologies are outlined. Some of the smaller frameworks and technologies were omitted due to them being of minimal use or because they were inconsequential to the project as a whole.

\subsection{Javascript + MithrilJS}
MithrilJS is a client side library for building single page web applications\cite{mithriljs}. It does well in performance-tests compared to AngularJS and React---two highly regarded and well established frontend frameworks---and the download-size is tiny in comparison\cite{mithril-framework-comparison}. It features a style of development that omits HTML, which is part of the reason why it was chosen by the developers. It also includes out-of-the-box features to prevent code-injections.

\subsection{Python + AsyncIO}
AsyncIO is a module for python that provides infrastructure for single-threaded concurrent applications\cite{python-asyncio}. It is used by cion in the web API, backend and catalyst components. Asynchronous frameworks waste less time due to, if used correctly, having little to none "blocking" code, for example when waiting for \acrfull{io}. This does not mean that the code runs faster, but it does mean that it runs more efficiently.
AIOHTTP is a framework for serving and consuming web-services asynchronously, which is heavy in \acrshort{io}. This makes it suitable for cion. Having an efficient framework for handling web-requests lets the web-API of cion handle more requests concurrently and decreases the need to run multiple instances of the application. 

\subsection{RethinkDB}
RethinkDB is a distributed and document-oriented database designed for real-time applications\cite{rethinkdb}. Reactive programming is a good match for real-time databases. It was already in use before this project was started, and it was selected due to real-time communication being a central part of the cion solution and because cion does not use relational data. 

\subsection{Caddy}
Caddy, or caddyserver, is an easy-to-use web server with great support for HTTPS\cite{caddyserver}. It is used by cion on all components exposing a web-service. Its purpose is to provide users with an easy way for cion to handle HTTPS on its own. Without Caddy, a user would need some sort of TLS termination proxy between cion and the open internet, though one could still opt for this if desired.

\section{Distribution of Roles}
As is already noted in the project plan, the two main components of the project are the front- and backend. Harald Floor Wilhelmsen had the frontend as his main responsibility, while Erlend Tobiassen had the backend as his. Kenan Mahic had the role of researching how to extend cion to support Kubernetes, as well as making sure the front- and backend components worked well together. Even though each individual had their main responsibility, everyone worked on every component by providing help, and doing development.

\section{Scientific Method}
\label{sec:sciencemethod}
Evaluating security in a meaningful way has been a longstanding debate in the security community. There are few metrics, they are hard to measure, and it is difficult to actually ascertain anything from the few data points gathered. This makes it difficult to assess from both a quantitative and qualitative approach. For this paper the team has decided to go for a hybrid approach. As is mentioned in chapter \ref{chap:theory}, looking at the attack surface of theoretical self-hosted versus externally-hosted configurations will be the primary way in which the authors will conduct their quantitative research.

The authors applied their domain knowledge gained through developing a \acrshort{CD} solution to create a list of resources that could be compromised. From this list of resources the type hierarchy in figure \ref{fig:th} was developed. Judged by the authors' domain knowledge, the list of attack classes was assigned a payoff based on how attractive each attack class is as a target. The attack surface measurement of each configuration was then calculated as the weighted sum:
\[
\sum n(S_i)\times w_i
\]
where $n(S_i)$ is the number of times attack class $i$ appeared in the configuration and $w_i$ is the payoff weight associated with attack class $i$. This gives a measurement of the attack surface suitable for comparing the configurations against each other.

The authors researched two configurations of a minimal \acrshort{CD} implementation, one where the solution is hosted locally with the environment servers, and one where the solution is hosted externally e.g. in a cloud based setup.

Qualitative results are created by interpreting the relative attack surfaces using deductive reasoning.

\subsection{Caveats}
\begin{itemize}
    \item The list of attack classes and their assigned payoff is not based on empirical fact, but the authors' knowledge of the systems and their implementation.
    \item The attack surface is measured of a minimal imagined \acrshort{CD} solution, any specific implementation of a \acrshort{CD} system, including the implementation by the authors, may have features that increase their attack surface.
    \item The results are only meaningful when compared to systems in terms of the same attack classes.
\end{itemize}