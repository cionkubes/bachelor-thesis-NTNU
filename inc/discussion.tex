\chapter{Discussion}
\label{chap:discussion}

\section{Scientific}
%Hvorfor holdt hypotesene, eller hvorfor holdt ikke hypotesene? 
The authors observed a quantifiable reduction in the attack surface, and thus an increase in security, when using a self-hosted \acrshort{CD} solution as compared to using an externally hosted solution. The risk is dependant on, and grows with, the number of environments managed by the \acrshort{CD} solution. In an external solution the risk per environment grows faster than in a self-hosted solution. A typical small scale team might have three environments; production, quality assurance and testing. In this case the attack surface of an external solution is 3.5 times larger compared to a self-hosted solution.

A large part of the attack surface consists of the services required for environment management and deployment triggers. In order for the services to work, the environments and the \acrshort{CD} solution needs to communicate with each other. This makes it impractical to avoid communication through the internet for externally hosted solutions; this in turn exposes the services to a plethora of attackers that otherwise would need physical access.

In order for the \acrshort{CD} solution to function it needs the ability to launch and restart applications running in the environments the solution is configured to manage. In the case of an externally hosted solution, the certificates granting this permission have to be given out to the third party hosting the solution. This is a huge concern, even though these certificates can be revoked, as these certificates are often synonymous with root access to the environments' servers.

%Drøft hvordan resultatene kan forstås i f forhold til eller som svar på problemstillingen.
The research problem posed that "\textit{Developers and businesses avoid \acrshort{CD} out of fear for applications hosted outside their control.}" Based on the results the authors are of the opinion that this fear is justified. As such there is a need for a \acrshort{CD} solution that can be self-hosted and that is open-source so that the solution can be audited.

\section{Engineering}
% TODO: why the refactoring was done
% TODO: explain the refactoring of some libraries?

% How did the end product turn out?
% Did the contractor get what they wanted?
% What requirements where fulfilled and what were not?
% Why did the results turn out as they did?
% What was good about how the features were implemented?
%    Write about how the webhooks feature opens up cion to be part of a larger chain of CI/CD. As in triggering other services to for example run integration tests after a deploy or similar.
% What was not so good?
All features specified in the requirements document where fulfilled. Some features and changes not specified in the requirements document were also implemented. These changes were namely a page in the web UI to add and configure environments, polishing of the UI on multiple pages, changes to existing features to make them easier to use and a log-view page that lets the user traverse and filter cion's deployment-logs.

The technologies used in the original autumn project were new to the team. This resulted in some code written in unnecessarily complex or inefficient ways. Both of the major changes mentioned in \nameref{sec:coderefactoring} were changed to add support for asynchronous programming. A lot of cleanup was done throughout this project to make the code cleaner and faster, and elements in the \acrshort{ui} was re-designed to create a consistent feel across the site. This makes the \acrshort{ui} easier to use, because the refactored pages function in similar ways. This causes fewer violations of a pillar of \acrshort{ux}; "\textit{The user should not be surprised.}" 

All features except the listed \textbf{"Post deploy behaviour"} were directly requested by the contractor. The post-deploy feature was implemented as \textit{webhooks}. It was implemented so that cion can trigger other services after deployment, this makes cion viable as one part of a larger \acrshort{CI/CD} ecosystem. 

An example setup where the webhooks-feature is a part of a larger \acrshort{CI/CD} solution is where staging-tests are run automatically after an update has been deployed by cion. In such an example, cion would deploy the newly built application to a staging-environment which would trigger a webhook to be fired on a service to run integration tests on the application in the staging environment. The service running the tests would then send a webhook back to cion telling it to deploy the application to the production-environment.

\section{Administrative}

%\begin{wrapfig}{r}{7.65cm}
%\epigraph{Give me five minutes to chop down a tree and I will spend the first two and a half sharpening my axe.}{Anonymous woodsman}
%\end{wrapfig}
\vspace{-1cm}\epigraph{Give me five minutes to chop down a tree and I will spend the first two and a half sharpening my axe.}{Anonymous woodsman}

%REMEMBER TO WRITE ABOOT TECHNOLOGY

%Hva ble bra på grunn av valgt prosess, fremgangsmåte og teknologi?
The chosen hybrid project management methodology allowed the team to work independently and to implement features the best way they saw fit, which resulted in a code base that the team was comfortable with. This sets up the project for future work better than if a more strict project methodology was used. 


Implementation of large features going smoothly can be attributed to planning-sessions being held within the team. One such planning-session produced the conclusion in appendix \ref{appendix:plansndecisions}. Hence the quote at the start of this section. These sessions helped save time down the line. Discussion and brainstorming helped the team exhaust bad ideas before they invested time for testing. 

A semi-strict style-checking tool was introduced to encourage consistency across the code base. It did not strictly enforce a code style limiting the developers' creativity, but simply warned programmers when they wrote something that did not conform to the agreed upon style. Introducing this after a large chunk of the code base was already written, the fall-project that created the base software that this product is based upon, did lead to a lot of hours being spent on refactoring. However having a more consistent code base resulted in hours being saved, due to making the reading of each other's code easier. 

%Hva ble ikke bra på grunn av valgt prosess, fremgangsmåte og teknologi?
The chosen frontend framework, MithrilJS, has a style of development that was unfamiliar to the team, and required a lot of time being spent on learning. The team is of the opinion that a lot of these hours were saved after significant knowledge of how the framework works was acquired, and it did set up the code base to be less complex and more extensible by the end of the project.

%Hva ble bra eller dårlig uavhengig av valgt prosess, fremgangsmåte og teknologi?

%Hvorfor endret ting seg utifra prosjekt plan

The original project plan had to be adjusted multiple times during the bachelor assignment. As is documented in administrative results some of the features were estimated poorly. Part of the reason the estimates were poor can be attributed to how the scope of some of the tasks changed throughout the project. Some tasks were thought to be easier or more complex than they turned out to be, which also affected estimates. The team also had forgotten to take into account mandatory classes during the winter period that ended up delaying the development. This coupled with health issues meant we had worked fewer hours than planned during parts of the project. This could have been better managed by having considered the mandatory classes at the time of making the project plan. A buffer should also have been added in our estimates so we would have extra hours in case of unforeseen circumstances or working on extra features.

\subsection{Impact}
%Dere skal også drøfte arbeidet i forhold til et helhetlig systemperspektiv. Sett resultatene inn i en samfunnsmessig og økonomisk, eventuelt også miljømessig, sammenheng.

For businesses avoiding \acrshort{CD}-solutions because of the security concerns raised by them being hosted in the cloud, an open source locally hosted \acrshort{CD}-solution can be very welcome. Having an implementation of \acrlong{CD} is sought after in today's world of IT, since it has the potential to increase productivity and reduce time-to-market. Having the ability to use such a tool but still maintain good security-habits can be a contradiction. If such a tool did not exist, every business would have to implement their own, which in turn would result in large amounts of resources being spent on the development of a tool instead of just setting up an already proven technology. This means that introducing this tool to the market could result in cost savings for many businesses. Using an already proven technology, especially one with open source code, is preferred over "reinventing the wheel" in most situations.

\subsection{Ethics}
%Analyser relevante etiskeproblemstillinger i forhold til resultatene fra arbeidet. Etikk-kompendiet fra 1.året kan være nyttig, se [Søraker 2013].
There are some ethical concerns regarding the relationship between the product developed as part of this assignment and this report. As part of the assignment the team has made an open-source self-hosted \acrshort{CI/CD}-solution called cion, and researches the security benefits in self-hosted solutions versus externally hosted solutions. The findings of the scientific part of this report conclude that self-hosted \acrshort{CD}-tools are more secure than externally hosted or cloud-based solutions. The authors therefor has a vested interest in promoting cion.


\subsection{Group Dynamic}
\label{subsec:groupdynam}
The teamwork has mostly been free of problems. Dividing the work into distinct parts requires each individual to deliver on their responsibilities. The alternative would be spreading the work across the entire group and working on the same parts together, in a work method such as pairwise programming. This would lessen the problem of one team member falling behind, and give fewer code defects. It would also be slower, and need more man-hours for the same amount of work. This is the primary reason the team chose to work independently, by weighing man-hours as more important than the risks associated by working alone. Since everyone in the group was familiar with each other from previous projects, the risk of someone slacking was not a concern. 

Some parts did not work that well. Since this is a continuation of a previous project by a group with one new addition, the gap in knowledge between the new addition and the existing group was substantial. This made it difficult for the new addition to work independently and take initiative.
